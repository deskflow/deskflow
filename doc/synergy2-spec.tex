\documentclass{article}
\title{Synergy 2 Specification}
\author{Nick Bolton}

\begin{document}

\maketitle

\section{Synopsis}

Synergy needs rewriting from scratch in order to provide a long term fix for 
bugs such as the ``sticky keys'' issue. It is difficult to fix bugs in the
current version of Synergy without causing regressions.

\subsection{Work in Progress}

This document is our first step towards designing version 2 of Synergy, so it's
very much a work in progress, and might take some time to complete (that is, 
if we ever run with it). Also, please feel free to correct my bad spelling. 

\section{Methodology}

Many developers before now have vowed to rewrite Synergy. Unfortunately though,
this has never materialised due to lack of motivation and momentum. To solve 
this problem, we need to use an Agile development methodology (such as Scrum).

By assigning a very small number of user stories to 1 month sprints, we can
ensure that releases are being constantly pushed out.

\subsection{Unit Testing}

Open source software should be easy for new developers to modify without too
much risk of breaking everything. Currently, applying patches is very risky, 
since it usually cause regressions. This problem can be solved by using TDD 
(test driven development), which will help us to spot instabilities early and
establish blame.

\subsection{User Stores}

The problem with defining features for Synergy is that on the surface it appears
to be a very simple tool with one primary requirement; to move the cursor 
between the screens of independant computers. This is a problem in more than one
way:

\begin{enumerate}
\item It makes people feel that there should be more features (there seldom
 should!).
\item It's difficult to split that one requirement into many user stories.
\end{enumerate}

In Scrum, the primary requirement as described above, would be called an
 ``Epic'', because it will absolutely have to span multiple iterations. This
document will attempt to carve up this Epic in to deliverable features. One
strategy would be to divide by operating system; i.e. first we support only
Windows.

\end{document}
