\section{Design}

\subsection{Scope}

With a tool as conceptually simple as Synergy, it's very easy to get carried
away with feature requests. Some features we should not plan to include:

\begin{enumerate}
  \item Dragging windows between screens.
  \item Remote audio sharing.
  \item Connecting over the Internet.
  \item Smart phone or tablet as server.
  \item Remote desktop integration.
\end{enumerate}

\subsection{GUI}

While version 1 is simple to setup, it can be a little difficult to
troubleshoot. As command-line tools go, the existing version is great, but
can be a little involved for some users. Often these users are from a Windows
or Mac background where tool configuration is usually done with an intuitive
GUI.

Version 1 was originally designed as a command line tool, then later on, several
GUI applications were invented to try and make configuration easier. QSynergy is
one of those GUIs that found a good balance of cross-platform support and user 
friendliness. However, like all Synergy GUIs, it is limited by the lack of 
interoperability with the underlying command line utility. So we should design 
version 2 with IPC to the GUI firmly in mind.

\subsection{Wizard}

While we should support the existing config file format, some would prefer a
setup wizard to simplify the process. The wizard should display immediately
after installation on all platforms (with the option to skip the wizard). When
the wizard starts, the user should choose whether the machine is a client or a
server (the wizard should explain the difference with a simple diagram).

\subsubsection{Server steps}
\begin{enumerate}
  \item Show instructions for starting client wizard(s).
  \item Show connected clients in a status list (as they connect).
  \item Allow user to advance once all clients have connected.
  \item Allow user to drag and drop clients on a grid.
  \item Save config to file (at custom location) when complete.
\end{enumerate}

Often, the user's hostname will not be found on their DNS server, and in such
a case we should detect this, and ask the user to use an IP instead.

\subsubsection{Client steps}
\begin{enumerate}
  \item Prompt the user for the server hostname or IP (shown in server wizard).
  \item If unable to connect, show troubleshooting tips.
\end{enumerate}

Troubleshooting steps might include asking the user to temporarily turn off
their firewall, try pinging, etc. This will reduce the number of troubleshooting
requests coming into the website and mailing list.
